In recent years the increasing amount of collected data led to the possibility on the one hand, but also to the requirement of big data analysis.
More and more data was being produced. Especially retailers and the sector of e-commerce are producing enormous amounts of data every day and have the urge to get aggregated information out of this data.
In order to cope with the growing data, extract programs were built that made the data usable for specific applications.
In this manner data was pulled and copied to other places and under the control of someone else.
Soon problem occured like Lack of data credibility, issues in productivity and the inability to transform data into information and a solution was needed to clear the mess.
Therefore the the idea of \emph{data warehouses} arose.
A data warehouse is an analytical database that is used as the foundation of a decision support system.
It is designed for large volumes of read-only data. providing intuitive access to information that will be used in making decisions.
Particularly it is designed to be a \qq{self-service} for analysts to generate information out of the data rather relying on IT specialists.

\subsection{The approach of data warehouses}

Moody \andothers propose in \cite{moody2000enterprise} an approach of how a data warehouse can be structured as well as the methodology how to transform an enterprise database into the proposed structure.
They mention several possible schemes, namely \emph{Flat}, \emph{Terraced}, \emph{Star}, \emph{Snowflake Scheme}, that are suitable for a data warehouse.
Especially they name the \emph{Star Scheme} as a generic model.

In contrast to conventional OLTP Enterprise models the star scheme based OLAP models have the following advantages.
While OLTP models are desinged in a normalized way to get rid of redundancy, the data in dimensionals models is highly redundand and denormalized.
Due to the introduced redundancy the system is way more performant because only a few join need to be made for a query.
Besides that the denormalized model improves the ease-of-use, because the small number of tables can be understood easily.
Using the metaphor of a dimensional cube, it is made easier to think of operations that need to be made in order to get the desired results.

\subsection{A tool to generate dimensional models}

At a point where there exists a lot of interesting data in an enterprise database a tool comes in handy that can transfer this data into a dimensional model.
According to this we implemented in this project a tool which transforms an arbitrary database scheme into a Star Scheme for a data warehouse.
This paper describes how we achieved that goal.
In chapter \ref{sec:theory} we will clarify the methodology proposed by \cite{moody2000enterprise}.
Section \ref{sec:implementation} gives detailed information about how we implemented the system, which algorithms we came up with and how the tool is designed.
In order to show how the tool performs there are some results and evaluation shown in section \ref{sec:evaluation}.

